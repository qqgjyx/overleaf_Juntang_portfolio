\ProjectEntry
{Unsupervised Segmentation in Hyperspectral Imaging}
{Summer Research \& Independent Study with Prof. Xiaobai Sun, Prof. Nikos Pitsianis, Dimitrios Floros}
{Python, SG-t-SNE-\textPi, Spectral Methods, Community Detection}
{
  \bitem{Studied precursor clustering and community detection methods, collecting over 5 methods and more than 10 datasets for hyperspectral imaging segmentation.}
  \bitem{Implemented and optimized SG-t-SNE-\textPi{} algorithm for dimensionality reduction preserving local and global structure.}
  \bitem{Utilized k-nearest neighbor graphs, Stochastic Graph t-SNE, and Parallel Clustering with Resolution Variation for unsupervised segmentation.}
  \bitem{Developed Python packages mheatmap and pysgtsnepi for HSI data processing, achieving 600+ GitHub stars community adoption.}
  \bitem{Worked with Python (scikit-learn), MATLAB, and Julia for implementation and validation.}
}
{assets/1000_hsi/00_.png}
{\extlink{https://www.qqgjyx.com/files/p00-HSI.pdf}{Poster}}
{\badge{HSI} \badge{Graph} \badge{Unsupervised Learning}}

\textbf{Technical Highlights:}
Hyperspectral images contain hundreds of spectral bands per pixel, creating extremely high-dimensional data that is challenging to segment without labels. This project leverages spectral graph theory to build a similarity graph where pixels are nodes and edges encode spectral similarity.

\textbf{Methodology:}
\begin{itemize}[leftmargin=1.2em, itemsep=0.1em]
  \item Construct k-nearest neighbor graph in spectral space using efficient approximate nearest neighbor search
  \item Apply SG-t-SNE-\textPi{} to embed the graph structure into 2D/3D space while preserving cluster structure
  \item Use community detection to identify coherent spectral regions corresponding to materials or land cover types
  \item Validate segmentations against ground truth using metrics like adjusted Rand index and normalized mutual information
\end{itemize}

\textbf{Key Findings:}
\begin{itemize}[leftmargin=1.2em, itemsep=0.1em]
  \item BlueRed consistently outperforms traditional baselines on HSI clustering
  \item Robust to spectral noise and spatial variability; produces coherent segments
\end{itemize}

\textbf{Impact:} 
The approach sets a stronger baseline for unsupervised HSI analysis with practical implications in agriculture, climate, remote sensing, and medical diagnostics. Future work: improve computational efficiency and extend to more HSI scenarios.