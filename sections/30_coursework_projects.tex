% ============================================================================
% COURSEWORK & ACADEMIC ACHIEVEMENTS
% ============================================================================

\section*{Academic Achievements \& Teaching}
\addcontentsline{toc}{section}{Academic Achievements \& Teaching}

% ============================================================================
% ACHIEVEMENT 1: Kaggle Competition
% ============================================================================

\addcontentsline{toc}{subsection}{Kaggle: Stanford RNA 3D Folding (Bronze)}
\ProjectEntry
{Stanford RNA 3D Folding (Kaggle Competition)}
{Bronze Medal (Top 8\%, 1500+ teams)}
{Python, Boltz-1, Protenix, TM-score, US-align}
{
  \bitem{Parsed CSV-format sequence and label data, generated YAML-format inputs, and handled data preprocessing including sequence redundancy and multi-conformation reference structures.}
  \bitem{Integrated and deployed a dual-model prediction pipeline combining Boltz-1 and Protenix for RNA 3D structure prediction.}
  \bitem{Configured cache and advanced diffusion parameters for optimal inference performance.}
  \bitem{Calculated TM-score using US-align, fused model outputs, corrected invalid coordinates, and generated compliant submissions.}
  \bitem{Achieved top 8\% finish among 1500+ teams in highly competitive international competition.}
}
{assets/cs316_ui_1.png}
{\extlink{https://www.kaggle.com/competitions/stanford-ribonanza-rna-folding}{Kaggle Competition} \quad \extlink{https://github.com/qqgjyx}{GitHub}}
{\badge{Kaggle Bronze} \badge{Top 8\%} \badge{RNA Folding} \badge{Deep Learning}}

\vspace{1em}

\textbf{Technical Highlights:}

RNA 3D structure prediction is a critical challenge in computational biology, with applications in drug discovery and understanding RNA function. This competition required predicting 3D atomic coordinates from RNA sequences.

\textbf{Methodology:}
\begin{itemize}[leftmargin=1.2em, itemsep=0.1em]
  \item \textbf{Data preprocessing:} Handled complex multi-conformation reference structures and sequence redundancy
  \item \textbf{Model ensemble:} Combined predictions from Boltz-1 (Google DeepMind) and Protenix models
  \item \textbf{Structure alignment:} Used US-align for computing TM-scores to evaluate prediction quality
  \item \textbf{Coordinate correction:} Implemented validation and correction for physically invalid atomic coordinates
  \item \textbf{Optimization:} Tuned diffusion parameters and caching strategies for computational efficiency
\end{itemize}

\textbf{Key Results:}
\begin{itemize}[leftmargin=1.2em, itemsep=0.1em]
  \item Bronze medal finish (top 8\% out of 1500+ international teams)
  \item Successfully deployed production-ready prediction pipeline
  \item Achieved competitive TM-scores on validation and test sets
  \item Demonstrated ability to integrate state-of-the-art AI models for scientific applications
\end{itemize}

\textbf{Impact:} This competition showcased the application of cutting-edge deep learning models to fundamental problems in structural biology. The techniques developed here have broad applicability to protein and RNA structure prediction tasks.

\newpage

% ============================================================================
% TEACHING 1: CS 521 Matrix, Graph, and Network Analysis
% ============================================================================

\addcontentsline{toc}{subsection}{Teaching Assistant: CS 521 Matrix, Graph, and Network Analysis}
\ProjectEntry
{Teaching Assistant: Matrix, Graph, and Network Analysis (CS 521)}
{Graduate Course with Prof. Xiaobai Sun}
{Python, MATLAB, Spectral Methods, Graph Theory}
{
  \bitem{Assisted in teaching graduate course covering Perron–Frobenius Theorem (PageRank), Graph Laplacian (Fiedler Vector), and spectral embedding.}
  \bitem{Led recitations and office hours to review assignments and clarify concepts; managed course Canvas site and code base.}
  \bitem{Provided Python implementations in addition to instructor's MATLAB code for improved accessibility.}
  \bitem{Graded homework and delivered guest lecture comparing embedding spaces and clustering methods.}
  \bitem{Received positive feedback from instructor and students for making course administration efficient and concepts accessible.}
}
{assets/cs201_ds_1.png}
{\extlink{https://sites.duke.edu/cs521/}{Course website} \quad \extlink{https://github.com/qqgjyx}{Code examples}}
{\badge{Graduate TA} \badge{Spectral Methods} \badge{Graph Theory}}

\vspace{1em}

\textbf{Course Topics Covered:}

\begin{itemize}[leftmargin=1.2em, itemsep=0.1em]
  \item \textbf{PageRank \& Perron-Frobenius:} Eigenvalue analysis for ranking and importance measures
  \item \textbf{Graph Laplacian:} Fiedler vector, spectral partitioning, and community detection
  \item \textbf{Spectral Embedding:} Dimensionality reduction preserving graph structure
  \item \textbf{Matrix Factorization:} SVD, NMF, and applications to data analysis
  \item \textbf{Network Analysis:} Centrality measures, clustering coefficients, graph metrics
\end{itemize}

\textbf{Teaching Contributions:}

\begin{itemize}[leftmargin=1.2em, itemsep=0.1em]
  \item Created Python implementations of key algorithms (complementing MATLAB originals)
  \item Delivered guest lecture on "Comparing Embedding Spaces and Clustering Methods"
  \item Developed supplementary materials connecting theory to practical applications
  \item Provided one-on-one mentoring during office hours for complex mathematical concepts
  \item Managed course logistics including Canvas site, assignments, and grading
\end{itemize}

\vspace{1em}

% ============================================================================
% TEACHING 2: MATH 302 Numerical Analysis
% ============================================================================

\subsection*{Teaching Assistant: Numerical Analysis (MATH 302)}
\addcontentsline{toc}{subsection}{TA: Numerical Analysis (MATH 302)}

\textbf{Instructor}: Prof. Dangxing Chen \quad \textbf{Duration}: Jan 2025 - Mar 2025

\textbf{Responsibilities}:
\begin{itemize}[leftmargin=1.2em, itemsep=0.1em]
  \item Provided support for instruction in numerical analysis topics: root finding, interpolation, numerical differentiation and integration
  \item Led weekly recitations on Python/MATLAB implementations of numerical methods
  \item Introduced supplementary material from CS 521 to deepen students' understanding
  \item Received positive feedback for making abstract methods accessible through coding demonstrations
\end{itemize}

\vspace{0.5em}

\textbf{Topics}: Newton's Method, polynomial interpolation, numerical quadrature, finite difference methods, numerical ODE solvers

\vspace{1.5em}

% ============================================================================
% TEACHING 3: MATH 101 Calculus
% ============================================================================

\subsection*{Teaching Assistant: Calculus (MATH 101)}
\addcontentsline{toc}{subsection}{TA: Calculus (MATH 101)}

\textbf{Instructor}: Prof. Dangxing Chen \quad \textbf{Duration}: Feb 2024 - May 2024

\textbf{Responsibilities}:
\begin{itemize}[leftmargin=1.2em, itemsep=0.1em]
  \item Assisted in teaching class of 40+ students covering derivatives and integrals
  \item Led weekly recitations reviewing lecture concepts, guiding problem-solving techniques, and facilitating group discussions
  \item Received positive feedback for strengthening students' foundational knowledge and fostering interest in mathematics
\end{itemize}

\vspace{0.5em}

\textbf{Topics}: Limits, derivatives, integration techniques, applications to physics and optimization

\vspace{1.5em}

% ============================================================================
% ACADEMIC COURSEWORK
% ============================================================================

\subsection*{Selected Coursework (Dean's List with Distinction)}

\textbf{GPA}: 3.8/4.0 \quad \textbf{Honors}: Dean's List with Distinction (24FA, 24SP), Dean's List (23FA)

\begin{description}[leftmargin=5em, labelwidth=4.5em, labelsep=0.5em]
  \item[A+ Courses:] Deep Learning, Machine Learning, Matrix/Graph/Network Analysis, Databases
  \item[Mathematics:] Numerical Analysis, Calculus, Linear Algebra, Probability \& Statistics
  \item[Computer Science:] Algorithms, Data Structures, Operating Systems, Computer Architecture
  \item[Applied:] Computational Biology, Signal Processing, Scientific Computing
\end{description}
