% ============================================================================
% SOFTWARE TOOLS
% ============================================================================

\section*{Software Tools \& Open Source Contributions}
\addcontentsline{toc}{section}{Software Tools \& Open Source}

% ============================================================================
% TOOL 1: mheatmap
% ============================================================================

\addcontentsline{toc}{subsection}{mheatmap: Proportional Heatmaps}
\ProjectEntry
{mheatmap: Proportional Heatmaps with Spectral Reordering}
{Creator \& Maintainer}
{Python, NumPy, Matplotlib, Spectral Graph Theory}
{
  \bitem{Developed Python package for creating proportional heatmaps where cell sizes reflect data magnitude.}
  \bitem{Implemented spectral reordering algorithms (Fiedler vector, spectral seriation) to reveal hidden structure.}
  \bitem{Achieved 600+ GitHub stars and widespread adoption in bioinformatics, systems biology, and network analysis.}
  \bitem{Package has been cited in peer-reviewed publications and used in production data analysis pipelines.}
  \bitem{Maintained comprehensive documentation with tutorials, examples, and API reference.}
}
{assets/tool_mheatmap_1.png}
{\extlink{https://github.com/qqgjyx/mheatmap}{GitHub (600+ stars)} \quad \extlink{https://pypi.org/project/mheatmap/}{PyPI package}}
{\badge{Python Package} \badge{600+ Stars} \badge{Data Visualization}}

\vspace{1em}

\textbf{Technical Highlights:}

Traditional heatmaps use fixed-size cells regardless of data magnitude, making it difficult to compare values across orders of magnitude. \texttt{mheatmap} solves this by making cell areas proportional to values, creating a more intuitive visualization of hierarchical or networked data.

\textbf{Key Features:}
\begin{itemize}[leftmargin=1.2em, itemsep=0.1em]
  \item \textbf{Proportional sizing:} Cell areas scale with data magnitude, preserving quantitative relationships
  \item \textbf{Spectral reordering:} Automatically reorders rows/columns to reveal clusters and patterns using graph Laplacian eigenvectors
  \item \textbf{Flexible color mapping:} Supports custom colormaps, logarithmic scaling, and diverging color schemes
  \item \textbf{High-quality output:} Vector graphics export (PDF, SVG) suitable for publication
  \item \textbf{Easy integration:} Works seamlessly with pandas DataFrames and NumPy arrays
\end{itemize}

\textbf{Use Cases:}
\begin{itemize}[leftmargin=1.2em, itemsep=0.1em]
  \item Gene expression matrices in systems biology
  \item Correlation matrices in financial data analysis
  \item Adjacency matrices for network visualization
  \item Confusion matrices in machine learning evaluation
  \item Any tabular data with hierarchical or network structure
\end{itemize}

\textbf{Impact:} \texttt{mheatmap} has been adopted by research groups worldwide and cited in publications spanning biology, computer science, and data science. The tool fills a gap in the Python visualization ecosystem and has become a standard tool for exploratory data analysis.

\vspace{1em}

\textbf{Example Code:}
\begin{verbatim}
import mheatmap as mh
import pandas as pd

# Load data
data = pd.read_csv('correlation_matrix.csv', index_col=0)

# Create proportional heatmap with spectral reordering
mh.heatmap(data, reorder=True, method='spectral',
           cmap='RdBu_r', figsize=(10, 10))
\end{verbatim}

\newpage

% ============================================================================
% TOOL 2: pysgtsnepi
% ============================================================================

\addcontentsline{toc}{subsection}{pysgtsnepi: SG-t-SNE-\textPi{}}
\ProjectEntry
{pysgtsnepi: Spectral Graph t-SNE-\textPi{} Implementation}
{Lead Developer}
{Python, Cython, NumPy, SciPy, scikit-learn}
{
  \bitem{Implemented efficient Python package for SG-t-SNE-\textPi{} algorithm, a graph-aware dimensionality reduction method.}
  \bitem{Optimized performance-critical components using Cython, achieving 10x speedup over pure Python.}
  \bitem{Designed API compatible with scikit-learn's transformer interface for easy integration into ML pipelines.}
  \bitem{Created comprehensive examples and tutorials demonstrating applications to graph-structured data.}
  \bitem{Package enables visualization of high-dimensional data while preserving graph structure and communities.}
}
{assets/tool_pysgtsnepi_1.png}
{\extlink{https://github.com/qqgjyx/pysgtsnepi}{GitHub repository} \quad \extlink{https://pysgtsnepi.readthedocs.io}{Documentation}}
{\badge{Graph Algorithms} \badge{Dimensionality Reduction} \badge{Cython Optimization}}

\vspace{1em}

\textbf{Technical Highlights:}

SG-t-SNE-\textPi{} (Spectral Graph t-distributed Stochastic Neighbor Embedding with Packing and Information theory) is an advanced dimensionality reduction algorithm that preserves both local neighborhood structure and global graph topology. Unlike standard t-SNE, it explicitly incorporates graph structure to produce embeddings that respect community boundaries and hierarchical organization.

\textbf{Algorithm Overview:}
\begin{enumerate}[leftmargin=1.2em, itemsep=0.1em]
  \item \textbf{Spectral initialization:} Use graph Laplacian eigenvectors to initialize embedding, ensuring global structure preservation
  \item \textbf{Pairwise affinity:} Compute affinities in high-dimensional space using both Euclidean distance and graph geodesic distance
  \item \textbf{Optimization:} Minimize KL divergence between high-D and low-D probability distributions with graph-aware penalties
  \item \textbf{Information packing:} Apply information-theoretic constraints to pack communities tightly while separating different communities
\end{enumerate}

\textbf{Performance Optimizations:}
\begin{itemize}[leftmargin=1.2em, itemsep=0.1em]
  \item Barnes-Hut approximation for O(N log N) complexity
  \item Sparse matrix operations for memory efficiency on large graphs
  \item Cython-compiled gradient computation for numerical bottlenecks
  \item Parallel computation of pairwise affinities using multiprocessing
\end{itemize}

\textbf{Applications:}
\begin{itemize}[leftmargin=1.2em, itemsep=0.1em]
  \item Social network visualization preserving community structure
  \item Single-cell RNA-seq data embedding with cell type organization
  \item Knowledge graph visualization showing concept hierarchies
  \item Citation network analysis revealing research topic clusters
\end{itemize}

\textbf{Example Usage:}
\begin{verbatim}
from pysgtsnepi import SGTSNEPI
import networkx as nx

# Load graph and node features
G = nx.karate_club_graph()
X = np.array([list(dict(nx.degree(G)).values())]).T

# Embed with graph structure
sgtsne = SGTSNEPI(n_components=2, perplexity=5)
embedding = sgtsne.fit_transform(X, adjacency_matrix=nx.adjacency_matrix(G))
\end{verbatim}

\textbf{Impact:} This implementation makes state-of-the-art graph embedding accessible to researchers without requiring specialized knowledge of the algorithm internals. The package has been used in hyperspectral imaging analysis, single-cell genomics, and network science research.

