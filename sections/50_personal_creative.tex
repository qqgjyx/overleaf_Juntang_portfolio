% ============================================================================
% PERSONAL & CREATIVE PROJECTS
% ============================================================================

\section*{Personal \& Creative Projects}
\addcontentsline{toc}{section}{Personal \& Creative}

\vspace{0.5em}

\textit{Beyond formal research and coursework, I enjoy building tools that blend automation, creativity, and data visualization. These projects showcase practical problem-solving and exploratory programming.}

\vspace{1em}

% ============================================================================
% PROJECT: RA Automation Tools
% ============================================================================

\subsection*{Resident Advisor Automation Suite}
\addcontentsline{toc}{subsection}{Resident Advisor Automation Suite}

\textbf{Context:} As a Resident Advisor (RA) at Duke Kunshan University (Aug 2024 - Present), I faced repetitive administrative tasks: scheduling duty rotations, tracking residence hall maintenance requests, creating door decorations for residents, and generating monthly reports. I automated these workflows using Python, serving residents across 3 years while handling 50+ incidents.

\vspace{0.5em}

\textbf{Tools Developed:}

\begin{description}[leftmargin=4em, labelwidth=3.5em, labelsep=0.5em]
  \item[Duty Scheduler:] Constraint satisfaction algorithm that generates fair rotation schedules respecting RA preferences and conflicts
  \item[Door Decor Generator:] Python script to scrape Reddit images and resident roster for automatic door decoration creation, used by fellow RAs
  \item[Maintenance Tracker:] SQLite database with web interface (Flask) for logging and tracking maintenance requests with priority levels
  \item[Monthly Reports:] Auto-generated PDF reports with statistics, charts (matplotlib), and narrative summaries
\end{description}

\vspace{0.5em}

\textbf{Technical Highlights:}

\begin{itemize}[leftmargin=1.2em, itemsep=0.1em]
  \item \textbf{Duty Scheduler:} Formulated as constraint satisfaction problem (CSP); used backtracking with forward checking to find valid schedules
  \item \textbf{Door Decorations:} Template system with Jinja2-style placeholders; batch generation from CSV of resident names
  \item \textbf{Maintenance Tracker:} RESTful API with authentication; automated email notifications when requests are resolved
  \item \textbf{Reports:} ReportLab for PDF generation; matplotlib for charts; Jinja2 for HTML templates
\end{itemize}

\vspace{0.5em}

\textbf{Impact:}
\begin{itemize}[leftmargin=1.2em, itemsep=0.1em]
  \item Reduced time spent on scheduling from 2 hours/month to 5 minutes
  \item Door decor script was shared with and adopted by fellow RAs across Duke Kunshan University
  \item Maintenance tracking improved response time and organization efficiency
  \item Demonstrated practical application of programming skills to solve real-world administrative challenges
\end{itemize}

\vspace{2em}

% ============================================================================
% WORK EXPERIENCE & RESEARCH
% ============================================================================

\subsection*{Professional Experience \& Additional Research}
\addcontentsline{toc}{subsection}{Professional Experience \& Additional Research}

\textbf{Product Analyst Intern, NTT Data} (Jul 2023 - Aug 2023, Wuxi, China)
\begin{itemize}[leftmargin=1.2em, itemsep=0.1em]
  \item Assisted in backend development and conducted literature reviews on LLMs and agentic systems
  \item Authored professional report on software-related industries in China, focusing on AI innovation
\end{itemize}

\vspace{0.5em}

\textbf{Banker Intern, Bank of Huaxia} (Feb 2024 - May 2024, Kunshan, China)
\begin{itemize}[leftmargin=1.2em, itemsep=0.1em]
  \item Investigated client businesses, conducted credit analysis and market research
  \item Drafted over 50 audit reports on local electronics companies and conducted industry research
\end{itemize}

\vspace{0.5em}

\textbf{Materials Research with Prof. Xiawa Wang} (Jan 2024 - May 2024, Duke Kunshan University)
\begin{itemize}[leftmargin=1.2em, itemsep=0.1em]
  \item Researched temperature-induced electronic, magnetic, and structural properties of emerging solid-state materials including Pb$_{10-x}$Cu$_x$(PO$_4$)$_6$O (LK-99)
  \item Utilized temperature-dependent X-ray diffraction, Raman spectroscopy, and DFT calculations
  \item Produced conference paper presented at MC17 (Materials Chemistry 17, Royal Society of Chemistry)
  \item Publication: "Analyzing temperature-induced phase transitions in Pb$_{10-x}$Cu$_x$(PO$_4$)$_6$O" (co-first author)
\end{itemize}

\begin{figure}[h]
\centering
\includegraphics[width=0.7\textwidth]{assets/placeholder_1600x900.png}
\caption{Example output from duty scheduler showing 4-week rotation matrix with RA assignments color-coded by preference satisfaction level. The algorithm ensures equal distribution of weekend and weekday duties while respecting conflict constraints.}
\end{figure}

\vspace{2em}

% ============================================================================
% CREATIVE: Data Art
% ============================================================================

\subsection*{Generative Data Art}
\addcontentsline{toc}{subsection}{Generative Data Art}

\textbf{Concept:} Exploring the boundary between scientific visualization and artistic expression by creating aesthetically compelling images from real datasets.

\vspace{0.5em}

\textbf{Examples:}

\begin{itemize}[leftmargin=1.2em, itemsep=0.1em]
  \item \textbf{Neural Network Weights:} Visualized CNN filter weights as abstract patterns; used dimensionality reduction (PCA, t-SNE) to create 2D/3D compositions
  \item \textbf{Audio Waveforms:} Converted music into polar coordinate spectrograms with artistic color gradients; printed as posters
  \item \textbf{Fractal Generation:} Implemented Julia set and Mandelbrot set renderers with custom color palettes; explored connections to dynamical systems
  \item \textbf{Geographic Data:} Created minimalist maps from OpenStreetMap data with stylized rendering (inspired by Stamen Design)
\end{itemize}

\vspace{0.5em}

\textbf{Tools Used:} Python (NumPy, Pillow, matplotlib, seaborn), Processing (Java), D3.js for web-based interactive visualizations

\vspace{0.5em}

\textbf{Philosophy:} Data visualization shouldn't just communicate information—it should evoke curiosity and aesthetic appreciation. By treating data as creative medium, we can engage broader audiences with scientific concepts.

\begin{figure}[h]
\centering
\includegraphics[width=0.6\textwidth]{assets/placeholder_1600x900.png}
\caption{Generative art piece: Neural network weight visualization. Each pixel represents a weight value from a trained CNN; colors mapped using custom perceptually-uniform colormap. The emergent patterns reflect the hierarchical organization learned by the network.}
\end{figure}

\vspace{1em}

\textbf{Exhibitions \& Sharing:}
\begin{itemize}[leftmargin=1.2em, itemsep=0.1em]
  \item Displayed in Duke's student art gallery (2024 Spring showcase)
  \item Shared on GitHub with reproducible code and tutorials
  \item Featured in Duke Computer Science department newsletter
\end{itemize}

