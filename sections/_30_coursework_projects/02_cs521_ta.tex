\ProjectEntry
{Teaching Assistant: Matrix, Graph, and Network Analysis (CS 521)}
{TA to Prof. Xiaobai Sun}
{Spectral/Graph, Python/MATLAB, Teaching}
{
  \bitem{Assisted in teaching graduate course covering Perron–Frobenius Theorem (PageRank), Graph Laplacian (Fiedler Vector), and spectral embedding.}
  \bitem{Led recitations and office hours to review assignments and clarify concepts; managed course Canvas site and code base.}
  \bitem{Provided Python implementations in addition to instructor's MATLAB code for improved accessibility.}
  \bitem{Graded homework and delivered guest lecture comparing embedding spaces and clustering methods.}
  \bitem{Received positive feedback from instructor and students for making course administration efficient and concepts accessible.}
}
{assets/3002_cs521_ta/00_.png}
{\extlink{https://courses.cs.duke.edu/fall24/compsci521/}{Course website} \quad \extlink{https://gitlab.oit.duke.edu/cs521_24fa/pcodes}{GitLab repository}}
{\badge{TA} \badge{Spectral Methods} \badge{Graph Theory}}

\textbf{Technical Highlights:}
\begin{itemize}[leftmargin=1.2em, itemsep=0.1em]
  \item PageRank \& Perron-Frobenius: Eigenvalue analysis for ranking and importance measures
  \item Graph Laplacian: Fiedler vector, spectral partitioning, and community detection
  \item Spectral Embedding: Dimensionality reduction preserving graph structure
  \item Matrix Factorization: SVD, NMF, and applications to data analysis
  \item Network Analysis: Centrality measures, clustering coefficients, graph metrics
\end{itemize}

\textbf{Methodology:}
\begin{itemize}[leftmargin=1.2em, itemsep=0.1em]
  \item Created Python implementations of key algorithms (complementing MATLAB originals)
  \item Delivered guest lecture on "Comparing Embedding Spaces and Clustering Methods"
  \item Developed supplementary materials connecting theory to practical applications
  \item Provided one-on-one mentoring during office hours for complex mathematical concepts
  \item Managed course logistics including Canvas site, assignments, and grading
\end{itemize}


