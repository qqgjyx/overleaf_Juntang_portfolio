% ============================================================================
% VISUALIZATION: Network Graph Embedding
% ============================================================================

\subsection*{Graph Embedding: Citation Network Structure}

\begin{figure}[h]
\centering
\includegraphics[width=0.75\textwidth]{assets/placeholder_1600x900.png}
\caption{2D embedding of academic citation network using SG-t-SNE-\textPi{} algorithm. Each point represents a paper; colors indicate research communities detected by Louvain algorithm. Links show citation relationships (directional: citing $\to$ cited). The embedding preserves both local citation patterns and global community structure, revealing interdisciplinary bridges between machine learning (blue), systems biology (red), and computational neuroscience (green).}
\end{figure}

\textbf{Technical Details:}
\begin{itemize}[leftmargin=1.2em, itemsep=0.1em]
  \item Network: 5,234 papers, 18,627 citation edges from arXiv CS/q-bio
  \item Features: TF-IDF vectors of titles and abstracts (300 dimensions)
  \item Embedding: SG-t-SNE-\textPi{} with perplexity = 30, learning rate = 200, 1000 iterations
  \item Community detection: Louvain algorithm with modularity = 0.68
\end{itemize}

\textbf{Insight:} Graph-aware dimensionality reduction reveals the intellectual structure of scientific fields. Papers form tight clusters within communities but also show "bridge" papers connecting different disciplines. This visualization helps identify emerging research areas and potential collaboration opportunities.


