% ============================================================================
% VISUALIZATION: EEG Time-Frequency Analysis
% ============================================================================

\subsection*{EEG Time-Frequency Dynamics During Sleep Transitions}

% \begin{figure}[h]
% \centering
% \includegraphics[width=0.9\textwidth]{assets/proj_eeg_1.png}
% \caption{Time-frequency spectrogram of EEG activity during wake-to-sleep transition. Generated using continuous wavelet transform (Morlet wavelet) in R with ggplot2. Note the characteristic decrease in beta activity (13-30 Hz) and increase in delta power (0.5-4 Hz) as the subject transitions from wake (left) to NREM sleep (right). The sharp spindle activity (12-15 Hz bursts) marks Stage 2 sleep onset.}
% \end{figure}

\textbf{Technical Details:}
\begin{itemize}[leftmargin=1.2em, itemsep=0.1em]
  \item Continuous wavelet transform with Morlet wavelet (frequency range: 0.5-40 Hz)
  \item Time resolution: 30-second epochs; frequency resolution: 0.5 Hz bins
  \item Normalized power spectral density (dB scale relative to baseline)
  \item Created using custom R pipeline with \texttt{eegkit}, \texttt{signal}, and \texttt{ggplot2}
\end{itemize}

\textbf{Insight:} Time-frequency analysis reveals transient dynamics that are invisible in traditional power spectrum plots. This visualization technique is essential for understanding non-stationary brain activity during cognitive tasks and sleep state transitions.


